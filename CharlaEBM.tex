\documentclass[10pt]{beamer}
\usetheme[progressbar=frametitle]{metropolis}
\includeonlyframes{}
%%%%%%%%%%%%%%%%%%%%%%%%%%%%%%%%%%%%%%%%%%%%%%%%%%%%%%%%%%%%%%%%%%%%%%%%%%%%%%%%
%					Preambulo
%%%%%%%%%%%%%%%%%%%%%%%%%%%%%%%%%%%%%%%%%%%%%%%%%%%%%%%%%%%%%%%%%%%%%%%%%%%%%%%%
\usepackage[spanish,activeacute]{babel}
\usepackage{xcolor}
\usepackage{color}
\usepackage{colortbl}
\usepackage{amsmath}
\usepackage{amssymb}
\usepackage{graphicx}
\usepackage{latexsym}
\usepackage{ucs}
\usepackage[utf8]{inputenc}
\usepackage{wrapfig}
\usepackage{siunitx}
\usepackage{times}
\usepackage{tikz}
\usepackage{verbatim}
\usepackage{multimedia}
\usepackage{hyperref}
\usepackage{thumbpdf}
\usepackage{wasysym}
\usepackage{pgf,pgfarrows,pgfnodes,pgfautomata,pgfheaps,pgfshade}
\usepackage{url}
\usepackage{empheq}
\usepackage{fancybox}
\usepackage{esint}
\usepackage{lipsum}
\usepackage{listings}
\usepackage{mathptmx}
\usepackage{helvet}
\usepackage{tikz}%
\usepackage{circuitikz}
\usepackage{csvsimple}
\usepackage{pgfplots}
\usepackage{multimedia}
\usepackage{media9}
\usepackage{proba}
\usepackage[absolute,overlay]{textpos}
\usepackage{bibunits}
\usepackage{tcolorbox}
% \usepackage[
%texcoord,
%grid, gridunit=mm,gridcolor=red!60,subgridcolor=green!60]%
% {eso-pic}
\usepackage[makeroom]{cancel}
\usepackage{epstopdf}
\epstopdfsetup{outdir=./}
\newcommand{\themename}{\textbf{\textsc{metropolis}}\xspace}
\title{Un Modelo Estocástico para la Reconstrucción de Masa Osea}
\subtitle{XIX EOBM/XIII ENBM}
\date{Octubre 12 2017}
\author{Saúl Díaz Infante Velasco}
\institute{CONACYT-Universidad de Sonora}
\titlegraphic{%
\hfill\includegraphics[height=1.0cm]{./figures/bar_background.png}
}
\metroset{block=fill}
\input{setup}
\defaultbibliography{CharlaBib}
\defaultbibliographystyle{abbrv}
\begin{document}
	\maketitle
 	\section*{Introducción}
		%
%
% Esta parte funciona con Okular bajo linux.
% Para Windows, comentar el siguiente bloque y
% habilitar lo que sigue y presentar con acrobad reader
%
%
%
% %%%%%%%%%%%%%%%%%%%%%%%%%%%%%%%%%%%%%%%%%%%%%%%%%%
\begin{frame}{Proceso de Remodelación en BMU}
	\begin{tikzpicture}[remember picture,overlay]
	  \node[anchor=south west, inner sep=0pt] at (current page.south west) {%
    \movie[%
	    height = \paperheight,%
	    width = \paperwidth,%
	    poster,%
			showcontrols]{}{./Video/oc-ob.mp4}%
      };
  \end{tikzpicture}
\end{frame}
%%%Para windows (guacala!!!) es un purrun con el acrobad reader%%%
%\begin{frame}{Proceso de Remodelación en BMU}
%	\includemedia[
%	width=1\linewidth,height=0.6\linewidth,
%	activate=pageopen,
%	addresource=oc-ob.flv,        %adjust
%	flashvars={source=oc-ob.flv}  %adjust
%	]{\frame{Click!}}{VPlayer9.swf}
%\end{frame}
		\input{./Introduccion/diagramaRemodelacion.tex}
		\input{./Introduccion/modeloKomarova.tex}
		\input{./Introduccion/modeloSilvia.tex}
		\input{./Introduccion/objetivo.tex}
 	%\section{}
%%%%%%%%%%%%%%%%%%%%%%%%%%%%%%%%%%%%%%%%%%%%%%%%%%%%%%%%%%%%%%%%%%%%%%%%%%%%%%%%
	\begin{frame}{Esquema de Charla}
   		\setbeamertemplate{section in toc}[sections numbered]
   		\tableofcontents[hideallsubsections]
 	\end{frame}
 	\section{Perturbación estocástica}
		\input{./Perturbacion/JustificacionRuido.tex}
		\input{./Perturbacion/Alternativas.tex}
		\input{./Perturbacion/perturbacion.tex}
 	\section{Propiedades de la solución}
		\input{./PropiedadesSolucion/positividad.tex}
 	\section{Resultados numéricos}
		\input{./ResultadosNumericos/resultadosNumericos.tex}
 	\section{Comentarios Finales}
		\input{./ComentariosFinales/comentarios_finales.tex}
\end{document}
